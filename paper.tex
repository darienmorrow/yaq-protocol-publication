\documentclass{article}
\usepackage[letterpaper, margin=1in]{geometry}

\title{The yaq Project: a Protocol for Scientific Instrumentation}
\author{Kyle F. Sunden,
  John C. Wright,
  Blaise J. Thompson}

\begin{document}

\maketitle

\section{Abstract}

\section{Introduction}

Instrumentation development is a key part of the scientific enterprise.
Scientists have always relied on their creativity and wit to assemble new instruments capable of carrying out new experimental procedures.
As the scientific industry has advanced, scientists are increasingly able to directly purchase pieces of their instruments.
A modern instrument might contain a motor from company A, a sensor from company B, a specialty light source from company C, and a totally custom robotic sample stage.
The brand new scientific instrument, doing a brand new experiment, is cobbled together from a few dozen pieces of hardware, some brand new and some reused.
The instrumental novelty arises from the creative way that all of those pieces are assembled and controlled in the context of the experiment.

Software integration can be a frustrating piece of the modern instrument development process.
Each individual component has its own drivers and interface.
A lack of standards means that each scientist must develop her own instrument control application.
Weeks can be spent just integrating one new component into an existing project.
Code reusability is typically poor, and technical debt grows quickly in academic and educational contexts.
Sientists may struggle to rapidly innovate on their experimental design when each hardware addition requires major software development.

Briefly, we
Minimum viable capabilities for desired protocol.
Focus on making client development easy.
\begin{itemize}
  \item Networked machines (subitem--multilingual)
  \item Unified interface (traits)
  \item Self describing
  \item Built on exisiting protocols
  \item Reusable
  \item Synchronous clients
  \item Portable, easy to install
  \item Interface agnostic. Not just serial. Certainly not just SCPI.
\end{itemize}
Well defined, concrete interface.
Static interface.

There are several large exisiting open source hardware interface projects, such as EPICS \cite{DalesioLR1991a}, TANGO \cite{tango-controls.org}, and pymodaq \cite{WeberSebastien2021a}.
These projects are impressive at facility scale, but in our experience these do not scale well to single-investigator lab environments.
We have aimed to keep yaq as simple as possible, with the goal of staying easy to manage at very small scales without dedicated staff.
Certainly yaq lacks features compared to these exisiting projects, as we will highlight.

There are exisiting open source hardware interface projects with a focus on simplicity [cite INSTRBUILDER etc].
To our knowledge none of these meet the minimal capabilities as outlined above.

Overview of this publication.
yaq is a protocol.
yaq is an implementation that runs as a stand-alone daemon.
yaq has a variety of clients.
yaq is not opinonated about orchestration layer.

\section{Daemons}

In yaq, each component of an instrument is supported by a tiny lightweight process that runs in the background of your computer: a daemon.
Since each component has its own process, each daemon can be developed separately.
Using yaq, a typical instrument might have several daemons each supporting a particular component.
Daemon A might support a motor, daemon B a sensor, and daemon C a light source.
To do an experiment, a control program (a "client") must send commands to each of these daemons.
Each of the daemons is a separate application running in its own process, and importantly the client is also a separate application.

The separation between each daemon and client makes the yaq framework less fragile than monolithic applications.
The yaq framework is language agnostic.
For example, a daemon written in Python might be controlled by a client implemented using LabVIEW.
In yaq, each component of an instrument can be developed and distributed separately.
For example, two very different instruments might happen to use the same temperature sensor.
Because the temperature sensor daemon is its own independent program, both instruments can benefit from the same daemon.
As yaq grows, the "ecosystem" of existing daemons means that future instruments become easier and easier to develop.

There are currently 62 daemons in the yaq project supporting at least 40 kinds of hardware.
Because yaq is protocol based, anyone can design and publish new daemons extending our hardware support.
A living list of all daemons and supported hardware can be found on the yaq website.

\section{Protocol}

yaq is built on top of Apache Avro RPC \cite{AvroSpecification}.
Avro RPC is a Remote Procedure Call framework which allows interprocess communication between a client application and a daemon \(the server\) which has direct hardware access.
Avro provides an agreed upon standard for serialization of data and method calls from a remote \(client\) process.
The protocols defined by Avro are extensible, allowing daemons to define their own data types and declare additional messages that can be processed by the daemon.
Each daemon provides an Avro Protocol file \(avpr\) which provides a complete list of the types and allowed messages.

While Avro RPC itself is agnostic to transport layer, yaq specifies TCP sockets as the transport layer.
TCP socket libraries are ubiquitous, allowing for clients (or daemons) to be implemented in a variety of languages if required.
The use of TCP also works across standard networks, allowing for simple multi-machine instruments.
yaq uses ``stateful'' Avro RPC to allow multiple requests without requiring additional handshaking.

In addition to the standard, yaq specifices schema for configuration options and for state that is persistent across restarts of a daemon.

Properties and traits in general.

\section{Traits}

Each component of a scientific instrument may have many parameters.
Similar components typically have similar parameters, but there may be incidental differences between each implementation.
For example, consider two different monochromators.
The driver for monochromator A may expose the function "set wavelength", while the driver for monochromator B exposes "set angle".
Any experimentalist wishing to switch from monochromator A to B must first go through their entire codebase to fix all of the small differences in the driver implementation.
yaq is designed using an (optional) compositional system that helps enforce consistency between different kinds of hardware.

yaq defines "traits": collections of behaviors which are logically grouped together.
Each trait specifies a set of configuration options, state entries, and methods that a daemon can implement.
Daemons, then, can be composed of multiple traits.
Clients can trust that daemons which implement a given trait will behave in similar ways.
For example, any daemon that implements the "is-position" trait must expose the methods "get\_position" and "set\_position".
It becomes easy to write generic clients that work with any daemon that has certain traits.
Of course, daemons are always allowed to implement additional configuration, state, and methods in addition to those which are implied by their traits.

\section{Clients}

yaq's RPC design invites the creation of multiple client libraries specialized for specific purposes.
Again, daemons can communicate simultaniously with multiple clients.
For a yaq instrument it is normal to have multiple clients running simultaniously, each focused on a particular purpose.
This modularity speeds development and increases overall flexibility of the system.
We have created a few client libraries which we will discuss here.

yaqc [CITE] is a very basic Python client for yaq.
yaqc is totally generic and aims to support any conceivable yaq RPC.
This library is easy to use directly in Python scripts or within the REPL, but it's also very useful as a dependency when building more specialized Python clients.

yaqc-qtpy [CITE] is a specialty Qt-based [CITE] GUI client.
yaqc-qtpy provides a special QClient object which uses Qt threading primatives to provide performant and safe access to yaq from a Qt environment.
In addition, yaqc-qtpy provides an entry point which automatically builds a GUI from arbitrary protocols using traits.
This is an invaluable tool which provides a ``free'' graphical user interface to any daemon.

Bluesky is an extensive open source control layer developed collaboratively across several user facilities \cite{AllanDanielB2019a}.

Bridging to larger open source control layers.
PyMoDaq \cite{WeberSebastien2021a}

\section{Case Studies}

yaq's simplicity and extensible nature has allowed us to use it in a broad variety of instrumentation designs.
We highlight a few such designs here.

The Landis Group at UW-Madison is currently working on a new type of flow reactor: the Wisconsin Quench Kinetics Reactor (WiQK).
This reactor incorporates several computer-controlled valves and syringe pumps as well as various sensors.
The WiQK is rapidly evolving as researchers continue to test and refine their design.
Only a few researchers are actively using the reactor during this prototyping stage.
The Landis Group has written very basic Python scripts utilizing yaqc to orchestrate hardware for their reactor.
These lightweight scripts can be extensively refactored as the hardware and orchestration strategy changes dramatically during WiQK development.
This approach ensures that the Landis Group is not limited by their ability to orchestrate hardware as they refine their reactor design.

The Stahl Group at UW-Madison created a custom reactor which monitors gasses being produced or consumed in the reaction head-space.  \cite{SalazarChaseA2021a}
This reactor incorporates a collection of sensitive pressure transducers and a single heating process value under computer control.
yaq daemons are used to interface with each sensor and the heater controller.
We were able to create a lightweight Qt-based GUI which offers an easy user experience to researchers.
By separating hardware interface code from GUI code, each piece of the system became easier to produce and maintain.
Taken together, these components have been a solid software experience for many users over several years.

The Wright Group at UW-Madison needs to orchestrate a large variety of hardware in multidimensional scans for their complex spectroscopy experiments \cite{MukamelShaul2000a, WrightJohnCurtis2011a}.
This need for exquisite hardware control has resulted in several prior attempts at ``home-built'' orchestration software \cite{CarlsonRogerJohn1988a, MeyerKentAlbert2004b, KainSchuyler2017a, ThompsonBlaiseJonathan2018a}.
Now, using yaq, the Wright Group has been able to move to Bluesky rather than inventing their own sophisticated control software ``from scratch''.
Moving forward, the Wright Group hopes to spend less energy developing control software and more energy developing creative spectroscopy experiments.

yaq is also being used in several smaller ways throughout UW-Madison Chemistry.
Excitingly, all of these research groups are able easily benefit from each-other's daemon developments.
This level of collaboration is new to us in the orchestration software space.

\section{Conclusions}

\clearpage
\bibliographystyle{unsrt}
\bibliography{references}

\end{document}
